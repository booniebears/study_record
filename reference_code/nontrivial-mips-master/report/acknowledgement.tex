\chapter{声明与致谢}

\section{版权声明}

本项目涉及的代码均在 GitHub 开放,相应的仓库列举如下(名称即为链接):

\begin{description}
    \item[\href{https://github.com/trivialmips/TrivialMIPS}{\color{blue} TrivialMIPS}] 初版的 CPU 设计(运行在清华大学 ThinPad 实验板上)
    \item[\href{https://github.com/trivialmips/nontrivial-mips}{\color{blue} NonTrivialMIPS}] 最终提交的 CPU 设计(运行在龙芯实验板上)
    \item[\href{https://github.com/trivialmips/TrivialMIPS_Software}{\color{blue} Software}] C++ 编写的裸机(Baremetal)程序,包括 TrivialBootloader 等
    \item[\href{https://github.com/trivialmips/u-boot-trivialmips}{\color{blue} U-Boot}] 移植的 U-Boot 引导程序
    \item[\href{https://github.com/trivialmips/u-boot-trivialmips}{\color{blue} uCore}] 移植的 uCore 操作系统
    \item[\href{https://github.com/trivialmips/linux-nontrivial-mips}{\color{blue} Linux}] 移植的 Linux 内核
    \item[\href{https://github.com/trivialmips/openssl}{\color{blue} OpenSSL}] 适配硬件 AES 加速功能的 OpenSSL 程序
    \item[\href{https://github.com/trivialmips/trivial-dashboard}{\color{blue} TrivialDashboard}] 使用 Qt 撰写的 Dashboard 演示程序
\end{description}

这些项目均遵循它们特定的开源许可证,某些目录中可能包含受版权保护的内容,使用它们意味着您知晓并愿意承担任何可能的法律责任。

本报告著作权归作者所有。您被允许在不作任何修改的情况下重新分发此文档;未经许可,您不得以任何方式复制、引用或演绎其中的任何内容。

\section{致谢}

本项目开发过程中得到了来自清华大学计算机系张宇翔同学的大力支持,我们在此表示衷心的感谢。此外,我们的指导教师陈康老师、刘卫东老师给我们提出了许多宝贵的建议,唐适之同学、王邈同学、刘家昌同学也向我们提供了帮助,在此一并向他们表示感谢。